% !TEX root = main.tex
\section{Introduction and Objectives}
\label{sect:introduction}
Research for handwritten digit recognition (HWDR) by machine learning (ML) techniques has
stayed active in the past several decades. The sustained interest in HWDR is primarily due to
its broad applications in bank check processing, postal mail sorting, automatic address reading,
and mail routing, etc. In these applications, both accuracy and speed of digit recognition are
critical indicators of system performance. In a machine learning setting, we are given a training
data set consisting of a number of samples of handwritten digits, each belongs to one of the ten
classes ${D_j , j = 0, 1, ..., 9}$ where class $D_j$ collects all data samples labeled as digit j. The HWDR
problem seeks to develop an approach to utilizing these known data classes to train a multi-
category classifier so as to recognize handwritten digits outside the training data. The primary
challenge arising from the HWDR problem lies in the fact that handwritten digits (within the
same digit class) vary widely in terms of shape, line width, and style, even when they are
properly centralized and normalized in size.
Classification of multi-category data can be accomplished by ML methods that are originally
intended for classifying binary-class data using the so-called one-versus-the-rest approach.
Alternatively, there are ML techniques that deal with multi-category data directly. A
representative method of this type is based on principal component analysis (PCA). The objective of this
laboratory experiment is to learn PCA as a technique for pattern recognition and apply it to the
HWDR problem.