% !TEX root = main.tex
\section{Introduction and Objectives}
\label{sect:introduction}
In this experiment, we build a multi-output linear model to predict energy efficiency of residential
buildings in terms of heating and cooling loads. The database, from which the model in question
learns, was created by A. Tsanas and A. Xifara in 2012 [R1] and has since been popular for
performance evaluation of various prediction as well as classification techniques [R2].
In [R1], an energy analysis using 12 different building shapes was carried out, where the buildings
differ with respect to a total of eight features including relative compactness, surface area,   wall
area,   glazing area, and glazing area distribution, and so on. The data set contains 768 samples,
each sample is characterized by a vector $x$ with 8 components which are numerical values of the
eight features mentioned above. Also associated with each sample is a 2-component output
vector y representing heating and cooling loads of the building. Here we take the term “output
vector” to mean a functional mapping from a set of eight features of a building as seen in a vector
$x$ to the building’s heating and cooling loads.
Clearly, we are dealing with a dataset of the form ${(x n , y n ), n = 1, 2, ..., N}$ with $x_n \epsilon R^{8\times1}$ ,
$y_n \epsilon R{2\times1}$ and $N = 768$. The objective of the experiment is to develop a 2-output linear model that
predicts heating and cooling loads for an ``unseen'' residential building characterized by a new
feature vector $x$.
In this experiment, the above data set was divided into two sets – one for training and the other
for testing. The train data include 640 samples and the associated outputs while the test data
include 128 samples and their outputs, and the division was done at random.