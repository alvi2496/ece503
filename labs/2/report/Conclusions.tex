% !TEX root = main.tex
\section{Conclusion}
\label{sect:conclusion}
From our analysis we can see that PCA and Euclidean distance work well for the recognition of the hand digit characters. But the PCA was initially introduced for dimension reduction. We can see from our experiment that the dimension of our train dataset in reduced a great deal using PCA and it contributes to the high speed execution of our program resulting in under 1 sec per 1,000 characters. This result justifies our choice of using PCA as the feature extraction technique. There are a lot to be done in term of classifying the dataset. The accuracy of our analysis can be higher with some other classification techniques such as Support Vector Machine(SVM).

\section{References}
[1] R. A. Fisher, "The use of multiple measurements in taxonomic problems”, Annual Eugenics,
vol. 7, part II, pp. 179-188, 1936. \\\relax
[2] UCI Machine Learning, http://archive.ics.uci.edu/ml, University of California Irvine, School of
Information and Computer Science. \\\relax
[3] LABORATORY MANUAL, ECE 403/503 - OPTIMIZATION for MACHINE LEARNING, Prepared by: Wu-Sheng Lu, Department of Electrical and Computer Engineering, University of Victoria.
   